% Search for all the places that say "PUT SOMETHING HERE".

\documentclass[11pt, fleqn]{article}
\usepackage{tikz, tkz-euclide, amsmath,textcomp,amssymb,geometry,graphicx,enumerate, mathrsfs, fixltx2e, algorithmicx, qtree, gensymb, amsthm}
\usepackage[T1]{fontenc}
\usepackage{subfig, empheq}
\usepackage[]{algorithm2e}
\usetkzobj{all}

\def\Name{Utsav Baral}  % Your name
\def\SID{25694452}  % Your student ID number
\def\Homework{1} % Number of Homework
\def\Session{Spring 2016}

\title{MATH 110 --Spring 2016 --- Homework \Homework\ Solutions}
\author{\large{\Name, SID \SID}\\\small{\parbox{0cm}{\begin{tabbing}\textbf{1.2 \#}1, 2, 7, 8, 10, 13, 17\\\textbf{1.3 \#}1, 3, 8, 12, 19, 25, 29\end{tabbing}}}}
\date{}
\markboth{MATH 53--\Session\ Homework \Homework\ \Name}{MATH 110--\Session\ Homework \Homework\ \Name}
\pagestyle{myheadings}

\newenvironment{qparts}{\begin{enumerate}[{(}1{)}]}{\end{enumerate}}

\textheight=9in
\textwidth=6.5in
\topmargin=-.75in
\oddsidemargin=0.25in
\evensidemargin=0.25in

\captionsetup{belowskip=12pt,aboveskip=4pt}

\usepackage{color}
\definecolor{myblue}{rgb}{.8, .8, 1}

\newlength\mytemplen
\newsavebox\mytempbox

\makeatletter
       
\newcommand{\itab}[1]{\hspace{0em}\rlap{#1}}
\newcommand{\tab}[1]{\hspace{.2\textwidth}\rlap{#1}}

\newcommand{\PartialD}[2]{\frac{\partial #1}{\partial #2}}
\newcommand{\RegularD}[2]{\frac{d #1}{d #2}}
\newcommand{\vectSpace}[0]{\mathbf{V}}
\renewcommand\qedsymbol{$\blacksquare$}

\begin{document}
\maketitle
\textbf{}

\section*{1.2}

\begin{itemize}
    \setlength\itemsep{5ex}
    \item [\textbf{1.}]
    \textit{\underline{Problem:} Label the following statements as True or False}
    \begin{itemize}
        \item[(a)] \textit{Every vector space contains a zero vector.}\\[1ex]
            \textbf{\boxed{TRUE.}} By the definition of the properties of vector sets, there must be a zero vector or else the set is not a vector set.\vspace{2ex}
            
        \item[(b)] \textit{A vector space may have more than one zero vector.}\\[1ex]
            \textbf{\boxed{FALSE.}} By contradiction, Suppose there are multiple values for zero, $u$ and $u'$. $u + u' = u$ and $u' + u = u'$ therefore $u=u'$ and our assumption that there are more than one zero is false.\vspace{2ex}
            
        \item[(c)] \textit{In any vector space, $ax = bx$ implies that $a = b$.}\\[1ex]
            \textbf{\boxed{FALSE.}} When $x=\vec{0}$ then $ax=bx\;\forall a,b \in \mathbb{F}$\vspace{2ex}
            
        \item[(d)] \textit{In any vector space, $ax = ay$ implies that $x = y$.}\\[1ex]
            \textbf{\boxed{FALSE.}} If $a = 0$, then $ax=ay\;\forall x,y \in V$\vspace{2ex}
            
        \item[(e)] \textit{A vector in $\mathbb{F}^n$ may be regarded as a matrix in $M_{n\times1}(\mathbb{F})$.}\\[1ex]
            \textbf{\boxed{TRUE.}} The single column of the matrix can be interpreted as a vector and vise versa because there is a one to one correspondence and both are members of a vector space.\vspace{2ex}
            
        \item[(f)] \textit{An $m \times n$ matrix has $m$ columns and $n$ rows.}\\[1ex]
            \textbf{\boxed{FALSE.}} It has $m$ rows and $n$ columns\vspace{2ex}
        
        \item[(g)] \textit{In $P(\mathbb{F})$ only polynomials of the same degree may be added.}\\[1ex]
            \textbf{\boxed{FALSE.}} Any degree polynomial may be added.\vspace{2ex}
            
        \item[(h)] \textit{If $f$ and $g$ are polynomials of degree $n$, then $f + g$ is a polynomial of degree $n$.}\\[1ex]
            \textbf{\boxed{FALSE.}} Counter example: $f = -2x^n$ and $g=2x^n$\vspace{2ex}
            
        \item[(i)] \textit{If $f$ is a polynomial of degree $n$ and $c$ is a nonzero scalar, then $cf$ is a polynomial of degree $n$.}\\[1ex]
            \textbf{\boxed{TRUE.}} $f$ has term $ax^n$, where $a\neq0$ so $cf$ will have term $(c\cdot a) x^n$ where $ca\neq0$\vspace{2ex}
            
        \item[(j)] \textit{A nonzero scalar of $\mathbb{F}$ may be considered to be a polynomial in $P(\mathbb{F})$ having degree zero.}\\[1ex]
            \textbf{\boxed{TRUE.}} Any polynomial, $f \in P(\mathbb{F})$, with degree $0$ is written as $f = a_0$, where $a_0 \in \mathbb{F}$. Thus, any non-zero scalar $k \in \mathbb{F}$ maps to the zero-degree polynomial $f=k$ and any zero-degree polynomial $f = k' : k' \in \mathbb{F}$ maps to the non-zero scalar $k'$ Establishing a bijective correspondence\vspace{2ex}
            
        \item[(k)] \textit{Two functions in $F(S, \mathbb{F})$ are equal if and only if they have the same value at each element of $S$.}\\[1ex]
            \textbf{\boxed{TRUE.}} That is the definition of equals.
    \end{itemize}
    
    \item [\textbf{2.}]
    \textit{\underline{Problem:} Write the zero vector of $M_{3\times4}(\mathbb{F})$}
        \begin{equation}
            \begin{bmatrix}
                \mathbf{0_{\mathbb{F}}} & \mathbf{0_{\mathbb{F}}} & \mathbf{0_{\mathbb{F}}} & \mathbf{0_{\mathbb{F}}} \\
                \mathbf{0_{\mathbb{F}}} & \mathbf{0_{\mathbb{F}}} & \mathbf{0_{\mathbb{F}}} & \mathbf{0_{\mathbb{F}}} \\
                \mathbf{0_{\mathbb{F}}} & \mathbf{0_{\mathbb{F}}} & \mathbf{0_{\mathbb{F}}} & \mathbf{0_{\mathbb{F}}}
                \nonumber
            \end{bmatrix}
        \end{equation}
    
    \item [\textbf{7.}]
    \textit{\underline{Problem:} Let $S = \{0, 1\}$ and $\mathbb{F} = \mathbb{R}$. In $F(S, \mathbb{R})$, show that $f= g$ and $f + g = h$, where $f(t) = 2t + 1$, $g(t) = 1 + 4t - 2t^2$, and $h(t) = 5^t+1$.} \\[2ex]\textit{\underline{Solution:}}\\
        $F$ is the set of all functions that map the set $\{0,1\}$ to the set $\mathbb{R}$ So the functions $f, g, h \in F(S, \mathbb{R})$ given to us can be thought of as any function that is defined for the input values of $0$ and $1$ To show that $f = g$ we show that f and g map the values $0$ and $1$ to the same values in $\mathbb{R}$, respectively:
            \begin{equation}
                \begin{split}
                    f(0) &= 2\cdot(0) + 1 = \mathbf{1},\; &&f(1) = 2\cdot(1) + 1 = \mathbf{3}\\
                    g(0) &= 1 + 4\cdot(0) - 2\cdot(0)^2 = \mathbf{1},\; &&g(1) = 1 + 4\cdot(1) - 2\cdot(1)^2 = \mathbf{3}\\
                    f(0) &= g(0) \wedge f(1) = g(1) && \therefore \boxed{f(t)=g(t) \text{ in } F(S, \mathbb{R})}
                    \raisebox{-2ex}{\qed}
                    \nonumber
                \end{split}
            \end{equation}
        
        Now to show that $f + g = h$ in $F(S, \mathbb{R})$, we repeat the same process as above for $(f+g)(t)$ and $h(t)$:
            \begin{equation}
                \begin{split}
                    (f+g)(0) &= 2,\;\; h(0) = 2\\
                    (f+g)(1) &= 6,\;\; h(1) = 6\\[2ex]
                    (f+g)(0) &= h(0) \wedge (f+g)(1) = h(1)\;\; \therefore \boxed{(f+g)(t)=h(t) \text{ in } F(S, \mathbb{R})}
                    \raisebox{-2ex}{\qed}
                    \nonumber
                \end{split}
            \end{equation}
    
    
    \item [\textbf{8.}]
    \textit{\underline{Problem:} In any vector space $\mathbf{V}$, show that, $(a + b)(x + y) = ax + ay + bx + by$ for any $x,y \in V$ and any $a,b \in \mathbb{F}$.}\\[2ex]\textit{\underline{Solution:}}\\
        Starting with $(a + b)(x + y)$, we first let $z = x + y$; we know that $z\in\vectSpace$ by the 1st property of vector spaces. We can rewrite our equation as $(a + b)(z)$. But by \textbf{Axiom 8}, we can rewrite this as $az + bz$. Replacing $z$ with our original relation, we transform our expression to $a(x+y) + b(x+y)$. And by \textbf{Axiom 7} this is the same as $ax + ay + bx + by$. Thus completing our proof.\qed
    
    \item [\textbf{10.}]
    \textit{\underline{Problem:} Let $\mathbf{V}$ denote the set of all differentiable real-valued functions defined on the real line. Prove that $\mathbf{V}$ is a vector space with the operations of addition and scalar multiplication defined in Example 3.}\\[2ex]\textit{\underline{Solution:}}\\
        Let $f$ be a fucntion $\in\mathbf{V}$
        
    
    \item [\textbf{13.}]
    {\underline{Problem:} Let $\mathbf{V}$ denote the set of ordered pairs of real numbers. If $(a1,a2)$ and $(b1, b2)$ are elements of $\mathbf{V}$ and $c \in \mathbb{R}$, define $(a1, a2) + (b1,b2) = (a1 + b1, a2b2)$ and $c(a1,a2) = (ca1,a2)$. Is $\mathbf{V}$ a vector space over $\mathbb{R}$ with these operations? Justify your answer.}
    
    \item [\textbf{17.}]
    \textit{\underline{Problem:} }
\end{itemize}

\section*{1.3}
\begin{itemize}
    \setlength\itemsep{5ex}
    \item [\textbf{1.}]
    \item [\textbf{3.}]
    \item [\textbf{8.}]
    \item [\textbf{12.}]
    \item [\textbf{19.}]
    \item [\textbf{25.}]
    \item [\textbf{29.}]
\end{itemize}

\end{document}
